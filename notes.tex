\documentclass{article}
\usepackage[utf8]{inputenc}
\usepackage{amsmath}
\usepackage{amsfonts}

\title{Solving Polynomial Equations}
\author{}
\date{}

\begin{document}

\maketitle

\section*{Solving Polynomial Equations}

Let's say we have this equation:
\begin{equation*}
    2x^3 - 5x^2 - 5 + 12 = 0
\end{equation*}

This is a polynomial of the 3rd degree (as we have $x^3$). We can simplify that by applying the \textbf{rational root theorem}, which works as follows:

The divisors of the constant term (the last number) are divided by the divisors of the leading coefficient (the first number); the results are the possible values of $x$. Also, they can be positive or negative.

First, the constant term needs to be a number alone, so in this case we sum $-5$ and $12$, which gives us $7$. 

In this case, the constant term is $7$ and the leading coefficient is $2$, then:
\begin{itemize}
    \item Divisors of 2: $1, 2$
    \item Divisors of 7: $1, 7$
    \item Possibilities: $\pm 1, \pm 7, \pm 0.5, \pm 3.5$
\end{itemize}

\subsection*{1st attempt (testing $x=1$):}
\begin{gather*}
    2 \cdot 1^3 - 5 \cdot 1^2 + 7 = \\
    2 - 5 + 7 = 4
\end{gather*}
As the result is not 0, this is not our number. Let's continue.

\subsection*{2nd attempt (testing $x=7$):}
\begin{gather*}
    2 \cdot 7^3 - 5 \cdot 7^2 + 7 = \\
    2 \cdot 343 - 5 \cdot 49 + 7 = \\
    686 - 245 + 7 = 448
\end{gather*}

\subsection*{3rd attempt (testing $x=0.5$):}
\begin{gather*}
    2 \cdot 0.5^3 - 5 \cdot 0.5^2 + 7 = \\
    2 \cdot 0.125 - 5 \cdot 0.25 + 7 = \\
    0.25 - 1.25 + 7 = 6
\end{gather*}

\subsection*{4th attempt (testing $x=3.5$):}
\begin{gather*}
    2 \cdot 3.5^3 - 5 \cdot 3.5^2 + 7 = \\
    2 \cdot 42.875 - 5 \cdot 12.25 + 7 = \\
    85.75 - 61.25 + 7 = 31.5
\end{gather*}

\subsection*{5th attempt (testing $x=-1$):}
\begin{gather*}
    2 \cdot (-1)^3 - 5 \cdot (-1)^2 + 7 = \\
    2 \cdot (-1) - 5 \cdot 1 + 7 = \\
    -2 - 5 + 7 = 0
\end{gather*}
Finally we have 0, so $-1$ is the number.

\section*{Applying Briot-Ruffini}

In Briot-Ruffini we do the following:
\begin{enumerate}
    \item We hold the first number.
    \item We multiply it by $x$ ($-1$ in this case).
    \item We sum the result with the next number, and we hold this result too.
    \item We follow from step 2 until the end, which should result in 0.
\end{enumerate}

The numbers we hold are the coefficients of the new formula, which is now a second-degree formula. Let's see, we have:
\begin{equation*}
    2x^3 - 5x^2 + 0x + 7 = 0
\end{equation*}
(We put a $0$ before $7$ because the number of terms is always the degree + 1).

\begin{itemize}
    \item Hold 2: $2 \cdot (-1) = -2$
    \item $-2 + (-5) = -7$ (Hold -7)
    \item $-7 \cdot (-1) = 7$
    \item $7 + 0 = 7$ (Hold 7)
    \item $7 \cdot (-1) = -7$
    \item $-7 + 7 = 0$
\end{itemize}

Finally, we have coefficients $2, -7$ and $7$. Our new second-degree formula is:
\begin{equation*}
    2x^2 - 7x + 7 = 0
\end{equation*}

\section*{Bhaskara Method}

The formulas are:
\begin{equation*}
    \Delta = b^2 - 4 \cdot a \cdot c
\end{equation*}
\begin{equation*}
    x = \frac{-b \pm \sqrt{\Delta}}{2 \cdot a}
\end{equation*}

Calculations:
\begin{equation*}
    \Delta = (-7)^2 - 4 \cdot 2 \cdot 7 = 49 - 56 = -7
\end{equation*}
\begin{equation*}
    x = \frac{-(-7) \pm \sqrt{-7}}{2 \cdot 2} = \frac{7 \pm 2.64i}{4}
\end{equation*}

\textbf{Result:} $1.75 \pm 0.66i$

\textbf{Full Result:}
\begin{itemize}
    \item Real root: $x = -1$
    \item Complex roots: $x = 1.75 + 0.66i$ and $x = 1.75 - 0.66i$
\end{itemize}

\section*{Newton's Method}

Sometimes the rational root theorem fails, so we can use Newton's Method:
\begin{itemize}
    \item $f(x) = 2x^3 - 5x^2 + 7$
    \item $f'(x) = 6x^2 - 10x$
\end{itemize}

Choosing $x_0 = 2$:
\begin{equation*}
    f(2) = 2(2)^3 - 5(2)^2 + 7 = 3
\end{equation*}
\begin{equation*}
    f'(2) = 6(2)^2 - 10(2) = 4
\end{equation*}

Applying the formula: $x_1 = x_0 - \frac{f(x_0)}{f'(x_0)}$
\begin{equation*}
    x_1 = 2 - \frac{3}{4} = 2 - 0.75 = 1.25
\end{equation*}

Next iteration with $x = 1.25$:
\begin{equation*}
    f(1.25) = 2(1.25)^3 - 5(1.25)^2 + 7 = 3.09
\end{equation*}
\begin{equation*}
    f'(1.25) = 6(1.25)^2 - 10(1.25) = -3.125
\end{equation*}
\begin{equation*}
    x_2 = 1.25 - \frac{3.09}{-3.125} \approx 2.24
\end{equation*}

The value went up, and $f(x)$ is getting farther from 0. We should test other values, like $-2$, until we converge to $-1$.

\end{document}